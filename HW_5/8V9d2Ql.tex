\documentclass{article}
\usepackage[utf8]{inputenc}
\usepackage[fleqn]{amsmath}
\usepackage{amssymb}
\usepackage{mathtools}

\usepackage[T1,T2A]{fontenc}
\usepackage[utf8]{inputenc}
\usepackage{natbib}
\usepackage{graphicx}
\usepackage[left=2cm,right=2cm,
    top=2cm,bottom=2cm,bindingoffset=0cm]{geometry}
\newcommand\tab[1][0.7cm]{\hspace*{#1}}
\newcommand\Gammad{\text{Gamma}}
\newcommand\myp{\mathbb P}
\title{Домашняя работа 5. Дедлайн: воскресенье 6.12 23:59}

\begin{document}

\maketitle
\textbf{Задача 1} (3 балла)
 Пусть $\{X_n\}_{n=1}^{\infty}$ — последовательность независимых случайных величин, причем
$X_n$ принимает значения $- \sqrt{n}$, $\sqrt{n}$ с вероятностями 1/2 каждое. Выполняется для этой последовательности закон больших чисел?

\textbf{Решение} \\
Найдём $\mathbb E X_n = -\sqrt{n} * 1/2 + \sqrt{n} * 1/2 = 0$.\\
$\mathbb D X_n = \mathbb E(X_n^2) - (\mathbb E X_n)^2 = n$.\\
Последовательность СВ $X_n$ сходится по вероятности к 0 тогда и только тогда, когда $\phi_{X}(t) \longrightarrow \phi_0(t)$ \\
$\phi_{X_n} = 1/2 * e^{itn^\alpha} + 1/2 * e^{-itn^\alpha}, где \alpha = 1/2$ \\
$X_i$ - НСВ, тогда $\phi_{X_n} = \sqcap_{k=1}^{n} \cos{\frac{tk^{\frac{1}{2}}}{n}}$\\
Теперь попробуем ограничить модуль нашей ХФ => ограничить произведение модулей.\\ Получим: $|\phi| \leq (\cos{t * 1/2 * 1/\sqrt{n}})^{n/4} = 
((1-\frac{t^2}{8n}+o(\frac{1}{n}))^{-\frac{8n}{t^2}})^{t^2/32} \longleftarrow e^{-\frac{t^2}{32}} < 1$. Получается, что ЗБЧ не выполняется.



\textbf{Задача 2} (2 балла)
Пусть $\xi_1, \xi_2, \dots$ -- последовательность независимых случайных величин,

$$\mathbb{P}(\xi_n = \pm2^n) = 2^{-(2n+1)}, \space \mathbb{P}(\xi_n = 0) = 1 - 2^{-2n}$$

\textbf{Решение} \\
Найдём матожидание: $\mathbb E \xi_n = 2^n * 2^{-(2n-1)} - 2^n * 2^{-(2n-1)} = 0$
И дисперсию: $\mathbb D \xi_n = \mathbb E \xi^2 - (\mathbb E \xi)^2 = 2^{-1} + 2^{-1} + 0 = 1$.
Получается, что ЗБЧ выполняется по теореме Чебышева.


\\
\textbf{Задача 3} (5 баллов)

Игральная кость подбрасывается до тех пор, пока общая сумма выпавших очков не превысит $700$. Оценить вероятность того, что для этого потребуется более $210$ бросаний; менее $180$ бросаний; от $190$ до $210$ бросаний. (показать все выкладки и получить конкретное число)

\textbf{Решение} \\
Найдем матожидание очков, выпавших за 1 бросок: $\mathbb E = \frac{7}{2}$ \\
Дисперсия: $\mathbb D = \frac{35}{12}$. \\
Рассмотрим $\frac{S_n - n * \mathbb E}{\sqrt{n * \mathbb D}}$. Эта величина имеет нормальное распределение. Из таблицы находим интересующее нас значение: ~ $0.08$
Аналогично и для других случаев \\
менее $180$: $1 - 0.0085 = 0.0015$ \\
от $190$ до $210$ : $= 0.85$

\\

\textbf{Задача 4} (5 баллов)

Известно, что вероятность рождения мальчика приблизительно равна $0,515$. Какова вероятность того, что среди $10$ тысяч новорождённых мальчиков будет меньше, чем девочек? (показать все выкладки и получить конкретное число)

\textbf{Решение} \\

Здесь нужно пользоваться формулой Муавра-Лапласа.

PS: Артём, у меня срочное дело появилось, сейчас 19.50 6 декабря, допишу все потом (( \\

Исходя из условия задачи, число мальчиков должно быть $\leq 4999$
То есть в теореме Муавра-Лапласа $a = 0, b = 0,4999, p = 0,515, n = 10.000$
$\mathbb P (a \leq \frac{S_n - np}{\sqrt{np(1-p)}} \leq b)$ [где $S_n = \sum_{k=1}^{10.000} \xi_k$] = $\Phi (b) - \Phi (a)$.
Смотрим в таблице значение $\Phi$. $\Phi(a) = 0,5; \Phi(b) \simeq 0,69$.\\
Тогда ответ получается: $0,69 - 0,5 \simeq 0,19$

\\

\textbf{Задача 5} (5 баллов)

Докажите

\begin{center}
    $\lim\limits_{n \to +\infty} \sum\limits_{\lfloor n/2 + \sqrt{n} \rfloor}^{n} C_n^k 2^{-n} = 1 - \Phi(2)$
\end{center}
\\

\textbf{Решение} \\
Вспомним про биномиальное распределение: $Bi(n, p) = C_n^k p^k*(1-p)^{n-k}$.\\
В нашем случае $p = 1/2$, поэтому $Bi(n, 1/2) = C_n^k \frac{1}{2}^n$.\\
Матожидание: $\mathbb E = n*p = n/2$.\\
Дисперсия: $\mathbb D = n*p^2$.\\


\textbf{Задача 6} (5 баллов)


Пусть $\xi_1, \xi_2, \dots$ -- последовательность независимых одинаково распределённых случайных величин с конечными дисперсиями. Для любого фиксированного вещественного $x$ найти предел.
\begin{center}
$\underset{n \to \infty}{\lim}\myp{(\xi_1 + \xi_2 + \dots + \xi_n < x)}$
\end{center}
\end{document}
