\documentclass{article}
\usepackage[utf8]{inputenc}
\usepackage[fleqn]{amsmath}
\usepackage{amssymb}
\usepackage{mathtools}

\usepackage[T1,T2A]{fontenc}
\usepackage[utf8]{inputenc}
\usepackage{natbib}
\usepackage{graphicx}
\usepackage[left=2cm,right=2cm,
    top=2cm,bottom=2cm,bindingoffset=0cm]{geometry}
\newcommand\tab[1][0.7cm]{\hspace*{#1}}

\newcommand\myp{\mathbb P}

\title{Домашняя работа 1}

\begin{document}

\maketitle
\textbf{Задача 1} (5 баллов)

Отрезок длины $a_1 + a_2$ поделён на две части длины $a_1$ и $a_2$ соответственно. $n$ точек последовательно бросаются на удачу на отрезок. Найти вероятность того, что ровно $m$ из $n$ точек попадут на часть отрезка длины $a_1$.

\textbf{Решение}

Суммарная длина отрезка  $= a_1 + a_2$. Тогда вероятность попадания в ту или иную его часть равна $\frac{a_1}{a_1 + a_2}$ или $\frac{a_2}{a_1 + a_2}$. По условию $m$ из  $n$ точек попадают в отрезок длины $a_1$, остальные $n-m$ попадают мимо. Назовем это событие "A". Найдём $\myp(A)$. $\myp(A) = (\frac{a_1}{a_1 + a_2})^m * (\frac{a_2}{a_1 + a_2})^{n-m}$.
\newline

\textbf{Задача 2} (2 балла)

На плоскость нанесены горизонтальные параллельные прямые на одинаковом расстоянии $a$ друг от друга. На плоскость наудачу бросается монета (круг) радиуса $R$ ($R < \frac{a}{2}$). Найти вероятность того, что монета не пересечёт ни одну из прямых.

\textbf{Решение}

Плоскость => координаты $x$ и $y$. Плоскость бесконечная. Рассмотрим "полоску" бесконечной длины и высоты $= a$. Куда может упасть центр монеты так, чтобы не пересекать прямые? Найдем "высоту" $h$ этого отрезка. $h = a - 2R$. Этот отрезок нужно бесконечно горизонтально продолжить на плоскости. Сведем задачу к нахождению вероятности попадания в кусок длины $h$ отрезка длины $a$. $\myp(Попадание в h) = \frac{h}{a} = \frac{a - 2R}{a}$.
\newline

\textbf{Задача 3} (3 балла)

На шахматную доску случайным образом ставятся два белых короля. Найти вероятность того, что эти два короля будут бить друг друга.

\textbf{Решение}

Всего 64 клетки. Будем рассмотривать разные варианты положения первого короля. Всего их 3: король стоит где-то в середине поля - 36 клеток; король стоит в углу - 4 клетки; король с краю, но не в углу - 24 клетки. Далее рассмотрим те варианты позиций второго короля, при которых он будет бить первого. 1 - 8 клеток вокруг 1-го короля, 2 - 3 клетки вокруг 1-го, и 3 - 5 клеток вокруг 1-го. Осталось посчитать вероятность. $\myp = \frac{36}{64} * \frac{8}{64} + \frac{4}{64} * \frac{3}{64} + \frac{24}{64} * \frac{5}{64} = \frac{36 * 8 + 4 * 3 + 24 * 5}{64 * 64} = \frac{420}{4096}$. Но каждая клетка поля учитывалась 2 раза, поэтому $\myp = \frac{210}{4096} = \frac{105}{2048}$
\newline

\textbf{Задача 4} (1.5 балла)

В $n$ ящиках размещают $2n$ шаров. Найти вероятность того, что ни один ящик не пуст, если шары неразличимы и все различимые размещения имеют равные вероятности.

\textbf{Решение}

Будем размещать каждый раз по 1 шару в новый пустой ящик, пока не останется пустых. $\myp = 1 * \frac{n-1}{n} * \frac{n-2}{n} = * ... * \frac{1}{n} = \frac{(n-1)!}{n^{n-1}}$. То есть при выборе ящика для $i-$го шара мы можем положить его в $n - i + 1$. Оставшиеся $n$ шаров можно разместить как угодно.
\newline

\textbf{Задача 5} (\textit{продолжение 4}) (1.5 балла)

Найти вероятность того, что заданный ящик содержит ровно $m$ шаров.

\textbf{Решение}

Осталось распределить $n$ шаров по $n$ ящикам. Вероятность отсутствия пустых ящиков посчитана в предыдущей задаче, поэтому найдем вероятность того, что при размещении $n$ шаров по $n$ ящикам в выбранном ящике окажется $m - 1$ шар. Это задача о "шарах и перегородках". $n$ ящиков $=> n - 1$ перегородок. Вероятность попадания $m - 1$ шара в какой-то из $n$ ящиков $= \frac{n}{n^{m-1}} = \frac{1}{n^{m-2}}$. Осталось разместить $n - m + 1$ шар по $n$ ящикам. Сколькими способами это можно сделать? Это будет $C_{2n-m}^{n-1} = \frac{(2n-m)!}{(n-1)!*(n-m+1)!}$. Можем дать ответ на задачу: $\myp = \frac{(n-1)!}{n^{n-1}} * \frac{1}{n^{m-2}} * \frac{(2n-m)!}{(n-1)!*(n-m+1)!} = \frac{1}{n^{n-1}} * \frac{1}{n^{m-2}} * \frac{(2n-m)!}{(n-m+1)!}$.
\newline

\textbf{Задача 6} (4 балла)

Пусть $A_1, A_2, \dots$ -- последовательность независимых событий. Доказать, что

\begin{center}
    $\myp(\bigcap\limits_{k=1}^{\infty} A_k) = \prod\limits_{k=1}^{\infty}\myp(A_k)$ 
\end{center}

\textbf{Решение}

События $A$ и $B$ независимы => $\myp(A|B) = \myp(A);\myp(AB) = \myp(A)\myp(B)$. 
В условиях задачи переобозначим $(\bigcap\limits_{k=2}^{\infty} A_k) = B$. Тогда имеем дело с выражением $\myp(A_1B) = \myp(A_1)*\myp(B)$, которое верно по определению. Дальше переобозначим $B$ как $A_2 * \bigcap\limits_{k=3}^{\infty} A_k = A_2 B$. (тут уже $B = \bigcap\limits_{k=3}^{\infty} A_k$). Получим результат $\myp(A_1)*\myp(A_2 * \bigcap\limits_{k=3}^{\infty} A_k) = \myp(A_1)*\myp(A_2) * \myp(\bigcap\limits_{k=3}^{\infty} A_k) = \prod\limits_{k=1}^{2}\myp(A_k) * \myp(\bigcap\limits_{k=3}^{\infty} A_k)$. Таким образом в итоге придем к выражению $\prod\limits_{k=1}^{\infty}\myp(A_k)$.
\newline


\textbf{Задача 7} (3 балла)

В самолете $n$ мест.
Есть $n$ пассажиров, выстроившихся друг за другом в очередь.
Во главе очереди -- <<заяц>> (пассажир без билета).
У всех, кроме <<зайца>>, есть билет, на котором указан номер посадочного билета.
Так как <<заяц>> входит первым, он случайным образом занимает некоторое 
место.
Каждый следующий пассажир, входящий в салон самолета, действует по такому 
принципу: если его место свободно, то садится на него, если занято, то занимает с 
равной вероятностью любое свободное.
Найдите вероятность того, что последний пассажир сядет на свое место.

\textbf{Решение}

Заяц не занимает место последего пассажира с вероятностью $\myp = \frac{n-1}{n}$. Следующий пассажир заходит и смотрит, не занято ли его место. Событие $A_i = $ [i-й пассажир не сел на место последнего], $H_i_1 = $ [его место занято], $H_i_2 = $ [его место свободно]. $\myp(A_i) = \sum\limits_{j=1}^{2}\myp(A|H_i_j)\myp(H_i_j)$. Дальше не сообржаю, куда прикручивать равновероятный выбор свободного места при занятом своём.
\newline


\textbf{Задача 8} (5 баллов)

Пусть мужик производит эксперимент, который может завершиться любым из $N$ способов, причем $i$-й результат происходит (независимо от мужика) с вероятностью  $p_i$. Пусть мужик может врать или говорить правду вне зависимости от того, какой результат наблюдает (хотя его ответ, естественно, от наблюдения зависит), причем говорит правду с вероятностью  $p_{true}$, а врет с вероятностью $p_{lie} = 1 - p_{true}$. Если он говорит правду, он называет результат, который имеет место. Если он врет, то он равновероятно говорит любой из оставшихся $N-1$  вариантов. Требуется найти вероятность того, что произошло условие $i$, при условии, что мужик сказал, что произошло условие $i$.

\textbf{Решение}

$\myp$(Произошло i|мужик сказал, что произошло i) $= \myp(p_i|p_{true}) = \frac{\myp(p_i p_{true})}{\myp(p_{true})}$


\end{document}
