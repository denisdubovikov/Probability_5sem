    \documentclass{article}
\usepackage[utf8]{inputenc}
\usepackage[fleqn]{amsmath}
\usepackage{amssymb}
\usepackage{mathtools}

\usepackage[T1,T2A]{fontenc}
\usepackage[utf8]{inputenc}
\usepackage{natbib}
\usepackage{graphicx}
\usepackage[left=2cm,right=2cm,
    top=2cm,bottom=2cm,bindingoffset=0cm]{geometry}
\newcommand\tab[1][0.7cm]{\hspace*{#1}}
\title{Домашняя работа 2 (дедлайн -- 17:00 8.10.19)}

\begin{document}

\maketitle
\textbf{Задача 1} (3 балла)

Пусть $X_1, \dots, X_n$ -- независимые одинаково распределенные случайные 
величины (н.о.р.с.в.) с функцией распределения $F(x)$.
Найти функции распределения $\max\limits_{1 \leq i \leq n} X_i$ и $\min\limits_{1 \leq i 
\leq n} X_i$.

\textbf{Решение}

$X_i$ - случайная величина, тогда $\max\limits_{1 \leq i \leq n} X_i$ и $\min\limits_{1 \leq i \leq n} X_i$ - случайные величины. Пусть $g(X_i) = \max\limits_{1 \leq i \leq n} X_i$, тогда $F_{\max\limits_{1 \leq i \leq n} X_i}(x)$ = $F_{g(X_i)}(x)$ = $P(g(X_i) < x)$. Аналогично и для $g(X_i) = \min\limits_{1 \leq i \leq n} X_i$.


\textbf{Задача 2} (3 балла)

Докажите, что функция $F_{\xi}(x) = P(\xi \leq x)$ непрерывна справа.

\textbf{Решение}

Непрерывность справа означает следующее: $\forall x \in \mathbb{R} \lim_{t\to x+0}F(t)$ = $F(x)$. Рассмотрим $P(t<\xi\leq x)$ = $P(\xi \in (-\infty;x] \cap \xi \in (t; +\infty))$ = $\dots$ = $1 - P(\xi \in (x; +\infty) \cup \xi \in (-\infty; t])$ = $1 - P(\xi \in (x; +\infty)) - P(\xi \in (-\infty; t])$ = $1 - 1 + F(x) - F(t)$ = $F(x) - F(t)$. Перейдём к пределу $\lim_{t\to x+0}$: $0 = \lim_{t\to x+0} P(t < \xi \leq x)$ = $F(x) - \lim_{t\to x+0} F(t))$.

\textbf{Задача 3} (3 балла)

Пусть $\xi_1, \xi_2, \dots, \xi_n$ -- независимые одинаково распределённые случайные величины, $P(\xi_k = i) = \frac{1}{N}$, $i = 1, 2, \dots, N$. Пусть $\eta_n = \xi_1 + \xi_2 + \dots + \xi_n$. Доказать, что $P(\eta_n$ делится на $n) \geq \frac{1}{N^{n-1}}$

\textbf{Решение}

Так как $\xi_1, \xi_2, \dots, \xi_n$ -- независимые одинаково распределённые случайные величины, то вероятность того, что все $\xi_i$ примут одинаковые значения a, где $a \in {1..N}$: $P(\xi_1 = a, \xi_2 = a, \dots, \xi_n = a)$ = $\frac{1}{N^n}$, так как с.в. независимые. Если в сумме $\eta_n = \xi_1 + \xi_2 + \dots + \xi_n \xi_1 = a, \xi_2 = a, \dots, \xi_n = a$, то $\eta_n$ делится на n. $a \in {1..N}$, поэтому таких значени а - N штук, тогда $P(\xi_1 = a, \xi_2 = a, \dots, \xi_n = a)$, где а уже любое число из {1..N} = $\frac{N}{N^n}$ = $\frac{1}{N^{n-1}}$. Но ведь мы можем получить $\eta_n$ кратное n не только путём сложения одинаковых чисел из {1..N}, но и путём сложения разных чисел из этого же диапазона, тогда искомое $P(\eta_n$ делится на $n) \geq \frac{1}{N^{n-1}}$



\textbf{Задача 4} (4 балла)

Допустим, что вероятность столкновения молекулы с другими молекулами в
промежутке времени $[t, t + \Delta t)$ равна $p = \lambda\Delta t+o(\Delta t)$ и не зависит от времени, прошедшего
после предыдущего столкновения $(\lambda = const)$. Найти распределение времени свободного
пробега молекулы (показательное распределение) и вероятность того, что это время превысит заданную величину $t^{\star}$

\textbf{Решение}

Рассмотрим событие $A_t$ = нет столкновений в промежутке (0;t].
$P(A_{t+\Delta t}|A_t)$ - условная вероятность. Нужно найти $P(A_t) = p(t)$, то есть вероятность того, что время свободного пробега не меньше.
$P(A_{t+\Delta t}|A_t) = 1 - P(not(A_{t+\Delta t})|A_t) = 1 - \frac{P(not(A_{t+\Delta t}))}{P(A_t)} = 1 - \frac{p(t+\Delta t)}{p(t)}$.
$A_{t+\Delta t} \subset A_t$ => $A_{t+\Delta t} A_t = A_{t+\Delta t}$.
Получаем $1 - \frac{p(t+\Delta t)}{p(t)}$ = $\lambda\Delta t+o(\Delta t)$. 
$\frac{p(t+\Delta t) - p(t)}{\Delta t} = -\lambda p(t) + o(\Delta t)$.
$\frac{dp(t)}{p(t)} = -\lambda dt$.
$p(t) = ce^{-\lambda t}, t > 0$;
$1 - p(t^*) = \lambda t^* + o(t^*)$;
$e^{-\lambda t^* = 1 - \lambda t^* + o(t^*)}$;
$1 - c + c\lambda t^* + o(t^*) = \lambda t^* + o(t^*)$;
Получаем $c = 1$.
$p(t) = e^{-\lambda t}$


\textbf{Задача 5} (2 балла)

Диаметр круга измерен приближенно. Считая, что его величина равномерно распределена в отрезке $[a, b]$, найти распределение площади круга, её среднее значение и дисперсию.

\textbf{Решение}

Диаметр круга - случайная величина $\xi$. Площадь - непрерывная возрастающая борелевская функция: $S \eta = f(\xi), f(\xi) = \pi\frac{\xi^2}{4}$ - тоже случайная величина. $F_{\eta}(y) = P(\eta < y) = P(f(\xi) < y) = P(\xi \in f^{-1}(-\infty; y))$. Существует $f^{-1}$ (тк непрерывна и возрастает). 
$F_{\eta}(y) = P(\eta < y) = P(f(\xi) < y) = P(\xi < f^{-1}(y)) = F_{\xi}(f^{-1}(y))$. 
Распределение нашли.
Мат ожидание: $\mathbb{E}_\eta = \int\limits_{-\infty}^{+\infty} \xi g(\xi)d\xi$, где $g(\xi): P(\xi \in [a, b]) = \int\limits_a^b g(\xi) d\xi$.
Дисперсия: $\mathbb{D}_\eta = \mathbb{E}(\eta - \mathbb{E}_\eta)^2$.

\textbf{Задача 6} (2 балла)

Пусть случайные величины $\xi$  и $\eta$ независимы и $\mathbb{E}\xi = 1$, $\mathbb{E}\eta = 2$, $\mathbb{D}\xi = 1$,  $\mathbb{D}\eta = 4$. Найти математические ожидания случайных величин:

а) $\xi^2 + 2\eta^2 - \xi\eta - 4\xi + \eta + 4$; б) $(\xi + \eta + 1)^2$

\textbf{Решение}
Раз $\xi и \eta$ независимые с.в., то $\mathbb{E}(\xi^2 + 2\eta^2 - \xi\eta - 4\xi + \eta + 4) = \mathbb{E}\xi^2 + \mathbb{E}2\eta^2 - \mathbb{E}\xi\mathbb{E}\eta - 4\mathbb{E}\xi + \mathbb{E}\eta + 4$, аналогично и для случая б).
Воспользуемся формулой $\mathbb{D}\xi = \mathbb{E}\xi^2 - \mathbb{E}^2\xi$.
Отсюда $\mathbb{E}\xi^2 = 2$.
Аналогично $\mathbb{E}\eta^2 = 8$
а) = 6
б) = 63

\textbf{Задача 7} (3 балла)

Пусть $\xi_1, \xi_2, \dots, \xi_n$ -- н.о.р.с.в. Доказать, что

$$\mathbb{E} \Big(\frac{\xi_1+\xi_2+\dots+\xi_k}{\xi_1+\xi_2+\dots+\xi_n\textbf{}}\Big)=\frac{k}{n}$$

дя любого $1 \leq k \leq n$.

\textbf{Решение}

$$\mathbb{E} \Big(\frac{\xi_1+\xi_2+\dots+\xi_k}{\xi_1+\xi_2+\dots+\xi_n\textbf{}}\Big) = \mathbb{E} \Big(1 - \frac{\xi_{k+1}+\dots+\xi_k}{\xi_1+\xi_2+\dots+\xi_n\textbf{}}\Big)$$


\textbf{Задача 8} (5 баллов)

На небольшом кластере GPU прямо сейчас очередь на обучение из 30 нейросетей. Единовременно на кластере может обучаться только одна сеть. За каждую нейросеть отвечают разные ML-инженеры. Нейросети имеют разные свойства: 10 из них больших (время их обучения 15 часов) и 20 маленьких (время их обучения 1 час). Пока не наступил момент начала обучения, разработчик переживает и внимательно следит за очередью, бесполезно растрачивая время. Посчитайте математическое ожидание, сколько человеко-часов будет потрачено на переживания разработчиков ровно с текущего момента (кластер освободился и начинает обрабатывать очередь, описанную выше), если задачи в очереди расположены в случайном порядке

\end{document}
