\documentclass{article}
\usepackage[utf8]{inputenc}
\usepackage[fleqn]{amsmath}
\usepackage{amssymb}
\usepackage{mathtools}

\usepackage[T1,T2A]{fontenc}
\usepackage[utf8]{inputenc}
\usepackage{natbib}
\usepackage{graphicx}
\usepackage[left=2cm,right=2cm,
    top=2cm,bottom=2cm,bindingoffset=0cm]{geometry}
\newcommand\tab[1][0.7cm]{\hspace*{#1}}
\newcommand\Be{\text{Be}}
\newcommand\Bi{\text{Bi}}
\newcommand\Po{\text{Poisson}}
\newcommand\Geo{\text{Geo}}
\newcommand\NB{\text{NB}}
\newcommand\Un{\mathcal U}
\newcommand\Exp{\text{Exp}}
\newcommand\N{\mathcal N}
\newcommand\Gammad{\text{Gamma}}
\newcommand\Beta{\text{Beta}}

\newcommand\myp{\mathbb P}
\newcommand\E{\mathbb E}
\newcommand\D{\mathbb D}
\newcommand\cov{\text{cov}}

\newcommand\dd{\text{d}}
\newcommand\II{\mathbb I}

\title{Домашняя работа 6}

\begin{document}

\maketitle
\textbf{Задача 1} (5 баллов)
Пусть $\xi_0, \xi_1, \dots$ -- норсв $\tilda U[0, 1]$. Найти плотность распределения
$$\eta_n = \prod\limits_{k=0}^{n}\xi_k$$

\textbf{Решение}

Так как величины независимые, $f(\xi_0, \xi_1 \dots) = f(\xi_0) f(\xi_1) \dots = 1$ (т.к. $f = 1$). \\
Рассмотрим величину $\eta = \xi_0 \xi_1$. (Для $\eta = \xi_0 \xi_1 \dots$ аналогично) \\
Функция распределения величины $\eta$ ($A$) будет следующей: $A = \iint\limits_{G} f(\xi_0, \xi_1) d\xi_0 d\xi_1$, где $G$ - множество, в общем случае $n$ -мерное, в котором лежит величина $\eta = \xi_0 \xi_1 \dots \xi_n$. \\
И тогда $A = \iint\limits_{G} f(\xi_0, \xi_1) d\xi_0 d\xi_1$, если $\eta \in [0; 1]$ и $A = 0$, если $\eta \in [0; 1]$

P.S. Ааааааа как все успеть мозг горит



\textbf{Задача 2} (5 баллов)
Пусть функции $f_1(x), f_2(x), f_3(x)$ удовлетворяют соотношению:
$$f_3(x) = \int\limits_{-\infty}^{\infty} f_1(x - u)f_2(u)du$$. Найти $f_1(x)$, если $f_2(x) = e^{-x^2}, f_3(x) = e^{\frac{-x^2}{2}}$


\textbf{Задача 3} (5 баллов)

Каждая целочисленная точка $k$ на числовой оси покрашена в белый цвет с 
вероятностью $p$ и черный с вероятнстью $q = 1 - p$ (независимо от остальных).
Пусть $B$ -- множество всех черных точек, а $S$ -- множество всех таких 
целочисленных точек $x$, что расстояние от $x$ до $B$ не меньше расстояния от 
$x$ до начала координат.
Найти математическое ожидание числа элементов множества $S$.


\textbf{Задача 4} (5 баллов)
Докажите, что для любых целых положительных $k$ и $n$ ($k \leq n$) справедливо неравенство:

$$C_n^k \leq 2^n\sqrt{\frac{2}{\pi n}}$$

\textbf{Задача 5} (5 баллов)
Пусть $\xi$ и $\eta$ -- независимые случайные величины с распределениями 
$\Beta(2, 1)$ и $\Exp(1)$ соответственно.
Найдите $\myp \left( \xi < \eta \right)$.

\end{document}
