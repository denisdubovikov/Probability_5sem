\documentclass{article}
\usepackage[utf8]{inputenc}
\usepackage[fleqn]{amsmath}
\usepackage{amssymb}
\usepackage{mathtools}

\usepackage[T1,T2A]{fontenc}
\usepackage[utf8]{inputenc}
\usepackage{natbib}
\usepackage{graphicx}
\usepackage[left=2cm,right=2cm,
    top=2cm,bottom=2cm,bindingoffset=0cm]{geometry}
\newcommand\tab[1][0.7cm]{\hspace*{#1}}
\newcommand\Gammad{\text{Gamma}}
\title{Домашняя работа 4}

\begin{document}

\maketitle
\textbf{Задача 1} (5 баллов)
Найти мат.ожидание и дисперсию для гамма-распределения.

\textbf{Задача 2} (4 балла)

Пусть $\xi$ с.в. с действительной характеристической функцией $f(t)$ и дисперсией $\sigma^2$. Доказать, что:
$$f(t) \geq 1 - \frac{t^2\sigma^2}{2}$$


\\
\textbf{Задача 3} (2 балла)

Случайные величины $\xi_1, \xi_2, \dots,\xi_n$ независимые случайные величины, причём $\xi_j \sim$
$\mathrm{N}(m_j, \sigma_j)$. Найти распределение случайной величины $\eta = \xi_1 + \xi_2 + ... +\xi_n$

\\
\textbf{Задача 3} (5 баллов)

Докажите, что сумма $n$ независимых случайных величин, равномерно 
распределенных на отрезке $[-1, 1]$, имеет плотность $f$, задаваемую формулой
\begin{center}
    

$f(x) = \frac{1}{\pi}  \int\limits_{0}^{+\infty} \left(\frac{\sin t}{t} \right)^n \cos(tx) dt$

\end{center}
Верна ли эта формула при $n = 1$?

\textbf{Задача 5} (4 балла)

Пусть $f$ - непрерывная, монотонно-возрастающая, неотрицательная, ограниченная функция, такая, что $f(0) = 0$.

Докажите, что для сходимости $\xi_n$ к 0 по вероятности необходимо и достаточно, чтобы сходилась к 0 последовательность $\mathbb E f(\vert \xi_n \vert)$

\\

\textbf{Задача 6} (5 баллов)
Пусть $\xi_1, \xi_2, \dots$ -- последовательность случайных величин с конечными дисперсиями. Положим $a_n = \mathbb E \xi_n$, $\sigma_n^2 = \mathbb D\xi_n$. Доказать, что если $a_n \rightarrow \infty$ и $\sigma_n^2 = o(a_n^2)$ при $n \rightarrow \infty$, то 

$$\frac{\xi_n}{a_n} \overset{P}{\rightarrow} 1, n \rightarrow \infty$$

\\

\end{document}
